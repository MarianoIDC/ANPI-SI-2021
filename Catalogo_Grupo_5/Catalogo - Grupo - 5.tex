\documentclass[10pt,a4paper]{article}
\usepackage[utf8]{inputenc}
\usepackage[T1]{fontenc}
\usepackage[spanish]{babel}
\usepackage{amsmath,multicol,enumerate}
\usepackage{amsfonts}
\usepackage[psamsfonts]{amssymb}
\usepackage{latexsym}
\usepackage{makeidx}
\usepackage[dvips]{graphicx}
\usepackage{multicol}
\usepackage{wrapfig}
\usepackage{color}
\usepackage[framed]{matlab-prettifier}
\usepackage[T1]{fontenc}
\usepackage{url}
\usepackage{listings}
\usepackage{geometry}

\renewcommand{\figurename}{Figura}
\renewcommand{\lstlistingname}{Código}

\definecolor{mygray}{rgb}{0.4,0.4,0.4}
\definecolor{mygreen}{rgb}{0,0.8,0.6}
\definecolor{myorange}{rgb}{1.0,0.4,0}

%%%%% Formato REVISTA DE MATEMÁTICA: TEORÍA Y APLICACIONES%%%%%%%
\topmargin=-2cm\textheight=23cm\textwidth=19cm
\oddsidemargin=-1cm\evensidemargin=-1cm
%%%%%%%%%%%%%%%%%%%%%%%%%%%%%%%%%%%%%%%%%%%%%%%%%%%%%%%%%%%%%%%%%
\parskip=0.25cm
\parindent=0mm
%%%%%%%%%%%%%%%%%%%%%%%%%%%%%%%%%%%%%%%%%%%%%%%%%%%%%%%%%%%%%%%%%

\author{Jonathan G Araya}
\title{Ecuaciones no Lineales}
\begin{document}
	%Encabezado
	Instituto Tecnológico de Costa Rica \hfill CE-3102: Análisis Numéricos para Ingeniería\\
	Ingeniería en Computadores \hfill Semestre: I - 2021\\
	
	
	\begin{center}
		\textbf{\huge Catálogo Grupal de Algoritmos}
	\end{center}
	
	{\bf Integrantes: }
	\begin{itemize}
		\item Josué Araya García - Carnet
		\item Jonathan Guzmán Araya - Carnet
		\item Mariano Muñoz Masís - Carnet
		\item Daniel Prieto - Carnet
	\end{itemize}
	
	\section{Tema 1: Ecuaciones no Lineales}
	
	\subsection{Método 1: Bisección}
	
	\lstinputlisting[style=Matlab-editor, basicstyle=\mlttfamily\normalsize, caption={Lenguaje M.}]{CodigosOctave/Biseccion.m}
	
	\lstinputlisting[language=Python, basicstyle=\ttfamily\normalsize, keywordstyle=\color{violet}\ttfamily, frame=single, commentstyle=\color{gray}\ttfamily, showstringspaces=false, caption={Lenguaje Python.}]{CodigosPython/Newton-Raphson.py}
	
	\lstinputlisting[basicstyle=\normalsize\ttfamily\color{black}, commentstyle=\color{mygray}, frame=single, numbersep=5pt, numberstyle=\tiny\color{mygray}, keywordstyle=\color{mygreen}, showspaces=false, showstringspaces=false, stringstyle=\color{myorange}, tabsize=2, language=C++, caption={Lenguaje C++.}]{CodigosC++/Biseccion.cpp}
	
\end{document}